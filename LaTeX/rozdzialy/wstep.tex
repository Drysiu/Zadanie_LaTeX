\chapter*{Streszczenie}
\addcontentsline{toc}{chapter}{Streszczenie}
Tekst analizuje rosnący trend migracji z miast na wieś, argumentując, że życie poza aglomeracją oferuje wyższą jakość codziennego funkcjonowania. Wskazane są cztery najważniejsze obszary, gdzie wieś jest lepsza: zdrowie, relacje społeczne, ekonomię oraz psychologię. Całość prowadzi do wniosku, że ucieczka z miasta jest świadomym wyborem na rzecz odzyskania równowagi życiowej i polepszenia psychiki.

\chapter{Wprowadzenie}
Miasto jest z natury bardziej atrakcyjne dla przeciętnego człowieka niż mieszkanie na wsi, ponieważ wszystko, czego potrzebuje, jest blisko – praca, edukacja, rozrywka. W ostatnich latach wsie i mniejsze miejscowości zyskują jednak coraz więcej mieszkańców, między innymi ze względu na możliwość pracy zdalnej.

Migracja ludności z miast na przedmieścia czy wsie spowodowana jest wzrastającymi cenami mieszkań, ucieczką od miejskiego hałasu, a także chęcią życia w spokoju i ciszy. Miasto nie jest dla każdego – ciągła pogoń, stanie w korkach czy wysokie koszty życia nie są zachęcające dla człowieka, który wolałby żyć spokojniej.

\begin{figure}[H]
    \centering
    \includegraphics[width=0.8\textwidth]{wies.png}
    \caption{Sielski krajobraz wiejski}
\end{figure}

Na wsi człowiek ma też znacznie bliższy kontakt z naturą. Zamiast betonu i spalin ma się lasy, pola i czyste powietrze. Jest to nie tylko przyjemniejsze, ale i zdrowsze – oznacza mniej alergii, niższy poziom stresu i naturalną motywację do spacerów czy jazdy na rowerze, bez konieczności szukania zatłoczonego parku.

Kolejną kwestią są relacje międzyludzkie. W mieście, mimo tłumów, łatwo o anonimowość i samotność. Na wsi społeczności są mniejsze i bardziej zżyte. Sąsiedzi się znają, pomagają sobie i tworzą poczucie wspólnoty, co daje większe bezpieczeństwo i oparcie w trudnych chwilach.

Ważna jest też sama przestrzeń. Za cenę małego mieszkania z widokiem na sąsiedni blok, na wsi można mieć dom z własnym ogrodem. Daje to zupełnie inną jakość życia – miejsce na grilla, przestrzeń do zabawy dla dzieci czy po prostu możliwość wypicia kawy na własnym tarasie w otoczeniu zieleni, a nie miejskiego zgiełku.